\documentclass{sig-alternate}
%\documentclass[10pt]{article}
%\clubpenalty=10000
%\widowpenalty = 10000
%\usepackage[margin=.9in]{geometry}
\usepackage[tight]{subfigure}

%\setlength\topmargin{0in}
%\setlength\headheight{0in}
%\setlength\headsep{0in}
%\setlength\oddsidemargin{0in}
%\setlength\evensidemargin{0in}
%\setlength\parindent{0.25in}
%\setlength\parskip{0in}

\usepackage{url}
\usepackage{multirow}
\usepackage{array}
\usepackage{epsfig}
\usepackage{footnote}
\widowpenalty=10000
\clubpenalty=10000
%\usepackage{setspace}
%\doublespacing

\begin{document}

\title{On the Design and Implementation of Structured P2PVPNs}

\numberofauthors{2}
\author{
\alignauthor
{David Isaac Wolinsky, Kyungyong Lee, Yonggang Liu, Oscar Boykin,
Renato Figueiredo}\\
       \affaddr{University of Florida}\\
       \email{\{davidiw, klee, yonggang, boykin, renato\}@acis.ufl.edu}
\alignauthor
{Linton Abraham}\\
  \affaddr{Clemson University}\\
  \email{labraha@clemson.edu}
}

\maketitle
\begin{abstract}
\begin{sloppypar}
In recent years, P2P VPNs have become quite popular by allowing users to connect
directly with each other bypassing the overhead of communicating through a third
party proxy. These P2P VPNs still require connecting to a central server for
authentication, NAT traversal, and proxying in the off chance NAT traversal
fails. While this significantly improves upon classical, centralized VPNs, this
approach now requires ether maintenance of all-to-all connections during
run-time or the involvement of a centralized authentication entity for each live
connection attempt.  For this solution, we propose a completely run-time
decentralized P2P model based upon a structured P2P system.  In this paper, we
will describe the components of this model and describe and evaluate our
reference implementation. A decentralized P2PVPN has an intuitive and simplistic
setup, reduces the requirements for connectivity, offers better proxy selection
in lieu of NAT traversal, and provides an opportunity for more intuitive trust
solutions. For evaluation, we will compare system and networking overheads of
the different VPN technology focusing on latency, bandwidth, CPU, and memory.
\end{sloppypar}
\end{abstract}

\section{Introduction}
A Virtual Private Network (VPN) provides the illusion of a Local Area Network
(LAN), namely direct communication, over a wide area network such as the
Internet while guaranteeing secure and authenticated communication amongst
participants.  Common uses of VPNs include accessing company or academic
network resources while traveling abroad, playing LAN based video games over the
Internet, connecting distributed resources from multiple sites, and securing
your Internet traffic while in unsecure locations.

In the context of this paper, the focus will be on VPNs that provide
connectivity between individual resources and so all resources that need
symmetric connectivity will need to be configured to the VPN.  While
traditional VPNs enable such distributed connectivity they do so at the
cost of maintaining a central server, which becomes the conduit for all traffic,
becoming a performance bottleneck and potentially removing end-to-end security.

To that end, there have been two directions 1) support for multiple VPN servers
for a single VPN\cite{openvpn} and 2) the use of P2P connections for bypassing central
communication that rely on run-time central authentication\cite{hamachi, wippien}.  In this paper, we
present a novel approach to this problem through the use of a fully run-time
decentralized P2PVPN (Peer-To-Peer).  P2P technology enables users to
communicate directly with each other bypassing the need for centralization
while providing all-to-all communication without maintaining all-to-all
connectivity with participants.  Interesting possibilities of P2P include
efficient wide area multicast, data distribution, storage, chat applications,
and even IP connectivity.

Current generation P2PVPNs do not provide features such as full-tunneling of
network traffic, such as forwarding Internet traffic, nor do they have efficient
mechanisms for multicast or broadcast.  P2PVPNs rely on direct connectivity
and in general will not work if NAT (Network Address Translation) traversal
between peers is unsuccessful.

The problems we seek to address with our P2PVPN model include:
\begin{itemize}
\small {
\setlength{\itemsep}{0pt}
\setlength{\parskip}{0pt}
\item fully decentralized run-time connectivity
\item full-tunneling for Internet traffic
\item intelligent relay selection in lieu of unsuccessful NAT traversal
\item efficient multicast and broadcast communication
}
\end{itemize}

A rudimentary overview of our solutions to the above problems follows and will
be covered in depth in the rest of this paper.  To provide fully decentralized
run-time connectivity, we use an automated certificate authority based
upon the use of user groups.  In the case of full-tunneling, P2PVPNs introduce
significantly more complexity since a simple routing table swap as done in
central VPNs no longer work, as such we investigate three different mechanisms
for tunneling all Internet traffic to our full-tunnel endpoint(s) besides our
P2P traffic.  When nodes cannot directly communicate, they seek to connect to
peers that are mutually physically close to each other and use them to relay
communication.  For efficient multicast and broadcast communication, we rely on
the use of bootstrapping a private P2P system whose members are only
participants of the VPN.

The rest of this paper is organized as follows.  Section II gives an overview
of current VPN technologies and the efforts to decentralized.  Section III
introduces P2P structures and our previous work IPOP (IP over P2P).  Section IV
describes the contributions of this paper, namely a feature-full P2PVPN.  In
Section V, we discuss our implementation and present evaluation comparing
centralized, P2P, and our VPN.  Finally, we give some concluding remarks in
Section VI.

\section{Virtual Private Networks}
\section{Peer-to-Peer Systems}
\section{Components of a P2PVPN}
\section{Evaluating VPN Models}
\section{Conclusions}

\bibliographystyle{abbrv}
\bibliography{nsdi10}
\suppressfloats
\end{document}
